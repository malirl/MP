\documentclass[a4paper,12pt]{book}


\usepackage[english]{babel}
\usepackage{blindtext}

%\usepackage[scaled=.92]{helvet}

\usepackage{microtype}
\usepackage{graphicx} 
\usepackage{wrapfig}
\usepackage{enumitem} 
\usepackage{amsmath} 
\usepackage{index}
%\usepackage{paralist}
%\usepackage{parskip}
%\usepackage[doublespacing]{setspace}
% \usepackage[latin1]{inputenc}
\makeindex



\begin{document}
\title{\Large{\textbf{Test}}}
\author{Adamofus}
\date{December 21, 2022}
\maketitle
\let\cleardoublepage\clearpage
\tableofcontents

	
\pagenumbering{roman}
\setcounter{page}{2}


\chapter{Projekce 3D do 2D}
\section{Perspektivní projekce}




\chapter{Vykreslení 3D scény}
\section{Stínění objektů}
\section{Vyplňování}







\chapter{Rotace}


Otáčet bod vůči středu soustavy souřadné je jako nanášet ho na otočenou soustavu souřadnou, tedy násobit vektory udávájící osy $x$, $y$, atd. takové soustavy.
Tyto vektory mají délku 1. Lze zapsat takto:

$p_n = \begin{pmatrix}
X_1 & Y_1 \\
X_2 & Y_2
\end{pmatrix}\vec{p}$\\


\section{2D}

Otočené souřadnice můžeme vyjádřit takto:


$X_1 = \cos(-\alpha) = \cos(\alpha)\\
X_2 = \sin(2\pi-\alpha) = -\sin(\alpha)\\
Y_1 = \cos(\pi/2 - \alpha) = \sin(\alpha)\\
Y_2 = \sin(\pi/2 - \alpha) = \cos(-\alpha) = \cos(\alpha)
$\\



$\begin{pmatrix}
\cos(\alpha) & \sin(\alpha)\\
-\sin(\alpha) & \cos(\alpha)
\end{pmatrix}\begin{pmatrix}x\\y\end{pmatrix}$



\section{3D}

Rotace bodu vůči ose Z ve směru hodinových ručiček:\\
$\begin{pmatrix}
\cos(\alpha) & \sin(\alpha) & 0\\
-\sin(\alpha) & \cos(\alpha) & 0\\
0 & 0 & 1
\end{pmatrix}\begin{pmatrix}x\\y\\z\end{pmatrix}$\\

Rotace bodu vůči zvolené ose:\\

Máme vstup osu $o$ a úhel $\alpha$. 

Mějme 2 soustavy souřadné $A$ a $B$ popsané jednotkovými vektory. Přitom $A$ je naše výchozí, na které vykreslujeme:

$\begin{pmatrix}
1 & 0 & 0\\
0 & 1 & 0\\
0 & 0 & 1
\end{pmatrix}$\\

Matice B má tvar:

$\begin{pmatrix}
a_x& b_x& o_x\\
a_y & b_y& o_y\\
a_z & b_z & o_z
\end{pmatrix}$\\

3. řádek $B$ tvoří souřadnice osy $o$, takže 3. souřadnice $p_b$ je právě vzhledem k $o$.

Čili je z následující rovnice zaručeno, že získáme přesně takový bod $p_b$, kde jeho 3. souřadnice je vzhledem k $o$. To potřebujeme, protože jsme zvolili matici
rotace podle osy Z (respektive 3. osy...), kterou chceme bod násobit. Takže tato souřadnice bodu v B po rotaci bude stejná.

Z rovnice $p_a = p_b B$ vyjádříme tedy $p_b$ vynasobením $Bˇ-1$: $p_b = p_a Bˇ-1$\\ %!


Zbývající vektory $a$ a $b$ v $B$ musí být jednotkové a vzájemně kolmé. takový $a$ dostaneme třeba ignorováním $z$ následovně: $\vec{a} = \{-y, x, 0\}$, kde $x$ a $y$ jsou souřadnice $o$. $b$ je už jen vektorovým součinem $a$ a $b$.

To je tedy převod relativních souřadnic ze soustavy $A$ do soustavy $B$.

% Prakticky nesejde na určení výchozího směru rotace.



Rotace probíhá následovně:\\

Vstup: osa otáčení $o$, úhel $\alpha$ 

\begin{enumerate}[label=\arabic*, font=\bfseries] % nummbered list
	\item převedeme souřadnice z A do B
	\item zrotujeme (například přes matici 1.1, tj. vůči 3. ose)
	\item vykreslujeme v A, takže převedeme z B do A
\end{enumerate}



\section{Posunutí středu a osy otáčení}

Pro posunutý střed v 2D platí, že vektor $\vec{XS}$, kde $S$ je střed a $X$ bod, má fakticky souřadnice bodu $\vec{p}$. K výpočtu stačí potom přičíst $S$.\\

Stejně u posunuté osy v 3D, je takový vektor vlastně bod $p_a$
V praxi, grafickém editoru, můžeme osu určit typicky dvěma body. A ten jeden z nich (třeba pro intuici první určený) je totiž tento střed $S$. Nakonec stačí opět přičíst $S$.






\end{document}



