\documentclass[a4paper,12pt]{book}


\usepackage[english]{babel}
\usepackage{blindtext}

%\usepackage[scaled=.92]{helvet}

\usepackage{microtype}
\usepackage{graphicx}
\usepackage{svg}
\usepackage{float}



\usepackage{wrapfig}
\usepackage{enumitem} 
\usepackage{amsmath, bm}
\usepackage{index}
\usepackage{algpseudocode}
%\usepackage{paralist}
%\usepackage{parskip}
%\usepackage[doublespacing]{setspace}
% \usepackage[latin1]{inputenc}
\makeindex


\graphicspath{ {./resources/infografika} }
\svgpath{resources/infografika/}



\begin{document}
\title{\Large{\textbf{Test}}}
\author{Adamofus}
\date{December 21, 2022}
\maketitle
\let\cleardoublepage\clearpage
\tableofcontents

	
\pagenumbering{roman}
\setcounter{page}{2}



\chapter{Projekce scény}
Projekcí scény je v této práci myšleno perspektivní vidění.
Paprsek jdoucí od promítaného bodu do oka pozorovatele se promítá na určenou rovinu v prostoru, tz. tvoří průnik s touto rovinou. Výsledný bod je přenesený na 2D soustavu souřadnou v této rovině.



\section{Stínění objektů v 3D}
\section{Zoom}




\chapter{Rasterizace}


Rasterizace je proces převodu vektorově definované grafiky do tzv. rastru, tedy mřížky skládající se z bodů (pixelů).
Takový rastr je základem obrazového výstupu na digitálních zařízení.
V následujícíh kapitolách jsou představeny algoritmy rasterizace 2 objektů: úsečky a kruhu.
Samostatná kapitola je pak věnovaná rasterizaci Beziérových křivek.


\section{Úsečka}



Převedení vstupu\\
Úsečka je část přímky definovaná dvěma body: $b_1$ a $b_2$.
Pro následující algoritmy platí, že souřadnice $x$ bodu $b_1$ je menší než $x$ bodu $b_2$, aby se mohly algoritmy posouvat o jeden dílek doprava, tj $x+1$.
Směrnice úsečky je $a = dy/dx$. 
Počítá se s vstupem $a\subset<0;1>$, takže úsečka svírá s osou $x$ úhel $0-45$ a $b_2$ je v 1. kvadrantu. Opět pro omezení na jedinou podmínku, zda je potřeba $y$ zvětšit či nikoli.


Tyto nároky umožňují zvolit efektivní algoritmus, ale současně vyžadují převod vstupu a výsledných souřadnic.\\

\begin{figure}[H]
  \centering
  \includesvg{fig1}
  \caption{Převod souřadnic podle směrnice úsečky.}
\end{figure}

%\includesvg[width=0.6\columnwidth](fg1.svg)



DDA\\
DDA využívá zaokrouhlované hodnoty $n*a$ pro výpočet $y_{new}$, je to prostý přístup. Současně ale zbytečně používá funkci zaokrouhlování či přetypování, kterou lze pro optimalizaci nahradit podmínkou s použitím proměnné,
protože jsou pouze 2 možnosti pro $y_{next}$: $y_{next} = y$, nebo $y_{next} = y+1$.

Proměnnou vyjadřuje $error(n)=((n*a)\mod1)-1/2$, kde $n$ je krok iterace, takže $error(0) = -1/2$.
Pokud platí zmíněná podmínka $error(n) >= 0$, platí současně $y_{next}=y+1$ a $error(n+1) = error(n) + a - 1$, jinak platí $error(n+1) = error(n) + a$.


Bresenhamův algoritmus\\
Další optimalizací je zbavení se desetinné čárky (tzv. float point number).
Představme si $error(n) = 2*dx*error(n)$. Jsou 2 možnosti:


$error(n+1)=error(n)+2*dy$\\
$error(n+1)=error(n)+2*dy-2*dx$

Nerovnice podmínky se nemění, protože po vynásobení pravé strany $2*dx$: $error(n+1)>=0*2*dx$ zůstává stejná.
Tím jsme se zbavili nutnosti použití desetinné čárky.




% tloustka cary, prostor, zdroje, infografika


\begin{algorithmic}
\State $x \gets x_1$
\State $y \gets y_1$
\While{$x<=x_2$}
\\vykresli bod[x,y]
\State $x \gets x + 1$
\State $error \gets error + 2d_y$
\If{error $\geq 0$}
    \State $y \gets y + 1$
    \State $error \gets error - 2d_x$
\EndIf 
\EndWhile
\end{algorithmic}





\section{Zrcadlení}

Zrcadlení je často využívaná operace, například pro rasterizaci kružnice.
Využívá bod $B$ a vektor $\vec{v}$, přes který $B$ zrcadlíme. Výstupem je zrcadlený bod $B_{z}$.
Platí, že vektor $\overrightarrow{BB_{z}}$ je kolmý na $\vec{v}$ a jeho délka je dvojnásobná vzdálenosti $B$ od $\vec{v}$.


$\frac{1}{2} \overrightarrow{BB_{z}} = (k*x+P_x;k*y+P_y)-(b_x;b_y)$, skalární součin tedy:
\\$(k*x+P_x-b_x;k*y+P_y-b_y)*(x;y)=0$.
\\Úpravou rovnice vyjde $k = -\frac{x(B_x-P_x)+y(B_y-P_y)}{x^2+y^2}$, bod získám:
$B_z = 2(k*\vec{v}+P)-B) = \bm{(2(k*x+P_x)-B_x;2(k*y+P_y)-B_y)}$



%$$




\section{Kruh} % Obdelník, trojúhelník, kružnice, elipsa, kvádr a koule

Pro kužnici rovněž existuje Bresenhamův algoritmus. Algoritmus vykreslí $1/8$ kružnice. V podmínce je tedy ze znalosti přímky svírající s osou $x$ 45 stupnu: $x>=y$. K vykreslení této části víme, že iterativně zvětšujeme $y$. $x$ dekrementujeme, pokud je hodnota $d$ kladná, tj. zajímají nás pouze 2 pixely. %infografika
Pro vykreslení celku stačí díky symetrii kružnice tuto část 3x zrcadlit.




\begin{algorithmic}
\State $d \gets x^2 + y^2 - r^2$
\State $x \gets r$
\State $y \gets 0$
\State $d \gets 0$
\While{$x>=y$}
\\vykresli bod[x,y]
\State $y \gets y + 1$
\State $d \gets d + d_y$
\If{$d \geq 0$}
    \State $x \gets x - 1$
    \State $d \gets d - d_x$
\EndIf 
\EndWhile
\end{algorithmic}

Jiný algoritmus se může zakládat na výběru z 3 sousedících pixelů podle toho, který je nejblíž středu. Takové řešení je ale neefektivní, implementací se zde nezabývám.






\section{Ostatní}

\section{Bezierova křivka}

\section{Vyplňování} % vyplňování jednoduchých tvarů a polygonů


Řádkovací metoda
\\Inverzní vyplňování
\\Flood fill


\chapter{Analýza}
\section{Detekce kolize}
\section{Oblasti průniku dvou objektů a průřez}
Sutherland–Hodgman algoritmus
\\Weiler–Atherton clipping algoritmus

\section{Obsah polygonu}
https://mathworld.wolfram.com/PolygonArea.html
\section{Je A v B?}




\chapter{Rotace}


Otáčet bod vůči středu soustavy souřadné je jako nanášet ho na otočenou soustavu souřadnou, tedy násobit vektory udávájící osy $x$, $y$, atd. takové soustavy.
Tyto vektory mají délku 1. Lze zapsat takto:

$p_n = \begin{pmatrix}
X_1 & Y_1 \\
X_2 & Y_2
\end{pmatrix}\vec{p}$\\


\section{2D}

Otočené souřadnice můžeme vyjádřit takto:


$X_1 = \cos(-\alpha) = \cos(\alpha)\\
X_2 = \sin(2\pi-\alpha) = -\sin(\alpha)\\
Y_1 = \cos(\pi/2 - \alpha) = \sin(\alpha)\\
Y_2 = \sin(\pi/2 - \alpha) = \cos(-\alpha) = \cos(\alpha)
$\\



$\begin{pmatrix}
\cos(\alpha) & \sin(\alpha)\\
-\sin(\alpha) & \cos(\alpha)
\end{pmatrix}\begin{pmatrix}x\\y\end{pmatrix}$



\section{3D}

Rotace bodu vůči ose Z ve směru hodinových ručiček:\\
$\begin{pmatrix}
\cos(\alpha) & \sin(\alpha) & 0\\
-\sin(\alpha) & \cos(\alpha) & 0\\
0 & 0 & 1
\end{pmatrix}\begin{pmatrix}x\\y\\z\end{pmatrix}$\\

Rotace bodu vůči zvolené ose:\\

Máme vstup osu $o$ a úhel $\alpha$. 

Mějme 2 soustavy souřadné $A$ a $B$ popsané jednotkovými vektory. Přitom $A$ je naše výchozí, na které vykreslujeme:

$\begin{pmatrix}
1 & 0 & 0\\
0 & 1 & 0\\
0 & 0 & 1
\end{pmatrix}$\\

Matice B má tvar:

$\begin{pmatrix}
a_x& b_x& o_x\\
a_y & b_y& o_y\\
a_z & b_z & o_z
\end{pmatrix}$\\

3. řádek $B$ tvoří souřadnice osy $o$, takže 3. souřadnice $p_b$ je právě vzhledem k $o$.

Čili je z následující rovnice zaručeno, že získáme přesně takový bod $p_b$, kde jeho 3. souřadnice je vzhledem k $o$. To potřebujeme, protože jsme zvolili matici
rotace podle osy Z (respektive 3. osy...), kterou chceme bod násobit. Takže tato souřadnice bodu v $B$ po rotaci bude stejná.

Z rovnice $p_a = p_b B$ vyjádříme tedy $p_b$ vynasobením $B^{-1}$: $p_b = p_a B^{-1}$\\ %!


Zbývající vektory $a$ a $b$ v $B$ musí být jednotkové a vzájemně kolmé. takový $a$ dostaneme třeba ignorováním $z$ a přehozením souřadnic následovně: $\vec{a} = \{-y, x, 0\}$, kde $x$ a $y$ jsou souřadnice $o$. $b$ je už jen vektorovým součinem $a$ a $b$.

To je tedy převod relativních souřadnic ze soustavy $A$ do soustavy $B$.

% Prakticky nesejde na určení výchozího směru rotace.



Rotace probíhá následovně:\\

Vstup: osa otáčení $o$, úhel $\alpha$

\begin{enumerate}[label=\arabic*, font=\bfseries] % nummbered list
	\item převedeme souřadnice z A do B
	\item zrotujeme (například přes matici 1.1, tj. vůči 3. ose)
	\item vykreslujeme v A, takže převedeme z B do A
\end{enumerate}



\section{Posunutí středu a osy otáčení}

Pro posunutý střed v 2D platí, že vektor $\vec{XS}$, kde $S$ je střed a $X$ bod, má fakticky souřadnice bodu $\vec{p}$. K výpočtu stačí potom přičíst $S$.\\

Stejně u posunuté osy v 3D, je takový vektor vlastně bod $p_a$
V praxi, grafickém editoru, můžeme osu určit typicky dvěma body. A ten jeden z nich (třeba pro intuici první určený) je totiž tento střed $S$. Nakonec stačí opět přičíst $S$.


\chapter{Generování tvarů a fraktály}









\end{document}



